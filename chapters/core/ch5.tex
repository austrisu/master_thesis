
\chapter{\textsc{Economic justification and analysis of socio-technical factors}}



The project is created to be open-source under MIT license and is not meant to be monetizing. As this system are available openly to organizations, companies or individuals will have the opportunity to replicate the project and train offensive IACS capabilities. This project can be found in GitHub repository frostyICS \footnote{frostyICS - IACS cyber range for offensive capability development (\url{https://github.com/austrisu/frostyICS})}. Time to create the research is estimated for 800 hours.


To calculate implementation cost, prices are acquired from publicly available sources. These prices could vary based on country. Price calculation is shown in table \ref{tab:cr-price} and the total hardware and software cost of one set is around 1 100 EUR. 


\begin{longtable}[c]{|c|p{0.7\textwidth}|r|}
	\caption{\raggedright{Price calculation of one set of CR.}}
	\label{tab:cr-price}\\
	\hline
	\textbf{Nr} & \textbf{Name}                                                           & \textbf{Unit price, EUR} \\ \hline
	\endhead
	%
	1           & SIMATIC S7-1200 Starter-Kit: CPU 1212C, STEP 7 Basic V16                & 700.00                   \\ \hline
	2           & IOT2040                                                                 & 210.00                   \\ \hline
	3           & LOGO! starter kit: LOGO! V8.3, LOGO! Soft Comfort, WinCC , power supply & 100.00                   \\ \hline
	4           & MS Windows 7 enterprise trial                                           & 0.00                     \\ \hline
	5           & NodeRed, open-source software                                           & 0.00                     \\ \hline
	\multicolumn{2}{|l|}{\textbf{}}                                                       & \textbf{1010.00}         \\ \hline
\end{longtable}



From a financial point of view, it is difficult to compare other created cyber ranges as there is a lack of such financial information in the research. Although IACS courses that claim to encompass cyber ranges can be found, and price ranges for them vary from a couple of thousands per person \parencite{WEB-25-sans-ics-course} to a couple of hundreds per year \parencite{WEB-26-ditechsolutions-ics-courses}. However, it is hard to determine the exact training scope and what type of cyber ranges are used. Based on that and the author concludes that the CR is somewhere in the middle of the price range. Hence, the price is reasonable considering the system's complexity. Moreover, from the author's experience, similar physical system automation usually costs more than ten times the CR cost.

Theoretical market volume can be estimated by looking at the need for offensive capability development exercise. As mentioned in section \ref{subsec:offensive-ops}, offensive capabilities are necessary for both private and government sectors. As IACS is getting more interconnected and more interesting for adversaries, governments will need to defend their critical infrastructure using offensive operations. Likewise, the private sector will need to conduct penetration tests on critical infrastructure to protect IACS from future threats. Thus, CR has the potential to be used by companies and government structures worldwide and, more specifically, in countries where Siemens products are strongly utilized.

Created CR has the potential for a positive influence on cybersecurity improvements in the IACS field. It will allow for offensive operation training and exercise. In that way, it is increasing understanding and awareness of adversary capabilities in the IACS segment. This CR being open source gives groups and individuals to increase skill sets. It also can shape the cybersecurity community by providing a chance to contribute to this project and improve it. 