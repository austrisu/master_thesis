\chapter*{\textsc{Annotation (ENG)}}
\addcontentsline{toc}{chapter}{\textsc{Annotation (ENG)}} 

Author: \me, Student number: \studentcode, \\
Supervisor: \supervisortitle \supervisor,  \\
Thesis title: \titleName.  

This thesis is written in the English language and has \pageref*{LastPage} pages, includes seven chapters, 12 tables, 16 figures and four annexes.

Industrial and automation control systems (IACS) are utilized, starting from manufacturing processes to energy transmission. As IACS  controls critical infrastructures, attacks on these systems can have devastating effects. Moreover, IACS systems are evolving by creating connections to conventional IT infrastructures what increase adversary access to industrial systems.

The research objective is to develop a realistic and easily reproducible IACS cyber range for offensive exercise development. 

The theoretical part of the work studies the concepts, trends, architecture, and scenarios of previously created IACS cyber ranges to create the basis for cyber range development. The author creates IACS cyber range in the practical part and uses it to conduct offensive capability development training. From the gathered information, the author's main conclusion is that IACS cyber ranges are a viable tool for understanding and trying tactics and techniques attackers may use to gain access to the IACS network and damage physical processes.
