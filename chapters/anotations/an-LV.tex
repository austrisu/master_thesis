\chapter*{\textsc{Anotācija (LV)}}
\addcontentsline{toc}{chapter}{\textsc{Anotācija (LV)}} 

Autors: \me, Studenta numurs: \studentcode, \\
Vadītājs: \supervisortitle \supervisor,  \\
Darba nosaukums: Industriālās un automātikas kontroles sistēmas kiberdrošības testēšanas laboratorija uzbrukuma spēju attīstīšanai.  


Šis maģistra darbs ir rakstīts angļu valodā un kopā satur \pageref*{LastPage} lapaspuses, septiņas nodaļas, 12 tabulas un 16 attēlus un četrus pielikumus. 


Industriālās un automatizācijas kontroles sistēmas (IAKS) vada un pārrauga dažādas sistēmas sākot no ražošanas procesiem līdz pat enerģijas pārvades tīkliem. Tādēļ, ka IAKS vada kritiskus procesus mūsu sabiedrībā, kiber uzbrukumiem, mērķētiem uz IAKS sistēmām, var būt katastrofālas sekas. Risku palielina arī tas, ka IAKS sistēmas attīstās un tiek integrētas ar tradicionālajām IT sistēmām, kas padara tās ar vien sasniedzamākas uzbrucējiem. 

Pētījuma mērķis ir izveidot reālistisku un viegli atkārtojamu IAKS kiberdrošības poligonu, kas palīdz attīstīt uzbrukuma spējas.
 
Lai radītu pamatu jaunas kiberdrošības laboratorijas izveidei, darba teorētiskajā daļā tiek apskatīti līdz šim izveidoto IAKS kiberdrošības laboratoriju tendences, arhitektūra un scenāriji. Praktiskajā daļā autors izveido IAKS kiberdrošības testēšanas poligonu, ko arī izmanto mācībās. Autora galvenais secinājums ir tāds, ka IAKS kiberdrošības poligons ir vērtīgs rīks, lai izprastu un izmēģinātu uzbrukuma veidus, ko uzbrucēji var izmantot IAKS tīklos, lai nodarītu kaitējumu industriālajiem procesiem.



